% !TeX encoding = UTF-8
% !TeX spellcheck = de_DE

%
% Ein ausführliches Beispiel zum Setzen der Kurzfassung zum
% GI-Dissertationspreis gemäß der Vorgaben unter
% https://gi.de/dissertationspreis
%
% Abgeleitet von
% https://github.com/gi-ev/LNI/blob/5ce2a334c17dd4496a1f56a469d5ce1fc84cee0c/lni-paper-example-de.tex
%
% Bei Fragen und Problemen:
% https://github.com/nm-wu/GIDiss-Beispiel/issues
%
% stefan.sobernig@wu.ac.at, Jan 2022
%

%% Dies gibt Warnungen aus, sollten veraltete LaTeX-Befehle verwendet werden
\RequirePackage[l2tabu, orthodox]{nag}

%% GIDiss: Als letzten Schritt, vor Einreichung, die Option
%% "draft" (verpflichtend!) entfernen. Alle anderen Optionen unberührt lassen, vor
%% allem "utf8" und "norunningheads"!

\documentclass[draft,utf8,biblatex,norunningheads]{lni}

%%%%%%%%%%%%%%%%%%%%%%%%%%%%%%%%%%%%%%%%%
%% GIDiss: Präambel (verpflichend)
%% GIDiss: NACH dieser Kommentarzeile nichts verändern! 
\usepackage{wrapfig}
\usepackage[backgroundcolor=yellow,obeyDraft=true]{todonotes}
\setuptodonotes{inline}
\makeatletter
\AtBeginDocument {
  \hypersetup{
    pdfinfo={
      Shorttitle={\@shorttitle}
    }
  }
 }
 \makeatother
 %% GIDiss: VOR dieser Kommentarzeile nichts verändern!
%%%%%%%%%%%%%%%%%%%%%%%%%%%%%%%%%%%%%%%%%

%% GIDiss: Zu Demonstrationszwecken (kann belassen, entfernt, oder
%% auskommentiert werden!)
\usepackage[math]{blindtext}
\usepackage{mwe}

%% LNI: Optionales
%% Schöne Tabellen mittels \toprule, \midrule, \bottomrule
\usepackage{booktabs}

%% BibLaTeX-Sonderkonfiguration,
%% falls man schnell eine existierende Bibliographie wiederverwenden will, aber nicht die .bib-Datei händisch anpassen möchte.
%% Bitte \iffalse und \fi entfernen, dann ist diese Konfiguration aktiviert.

\iffalse
\AtEveryBibitem{%
  \ifentrytype{article}{%
  }{%
    \clearfield{doi}%
    \clearfield{issn}%
    \clearfield{url}%
    \clearfield{urldate}%
  }%
  \ifentrytype{inproceedings}{%
  }{%
    \clearfield{doi}%
    \clearfield{issn}%
    \clearfield{url}%
    \clearfield{urldate}%
  }%
}
\fi

%%%%%%%%%%%%%%%%%%%%%%%%%%%%%%%%%%%%%%%%%
%% GIDiss: Titel- und Autoreninfos auf jeden Fall in der Präambel (vor
%% \begin{document} setzen!
%% Die Fußnote enthält die Adresse (Affiliation) sowie eine E-Mail-Adresse.
\title[Kurztitel (optional, max. 65 Zeichen)]{Sehr langer Titel der Dissertation bzw. der Kurzfassung in
  deutscher Sprache (ggfs. mehr als 65 Zeichen)\footnote{\todo[inlinewidth=.98\linewidth]{Wenn die Dissertation auf Englisch
      verfasst wurde, Originaltitel anführen!}Englischer Titel der Dissertation: ``Original title in English''}}
\author[Vorname Nachname]
{Vorname Nachname\footnote{Universität, Abteilung, Straße,
    Postleitzahl Ort, Land \email{vornamen.nachname@example.org}}}
%%%%%%%%%%%%%%%%%%%%%%%%%%%%%%%%%%%%%%%%%

%% biblatex erwartet dies bei der Präambel
\bibliography{gidiss-kurzfassung}

\begin{document}

\maketitle

\todo{Falls der deutsche Titel mehr als 65 Zeichen umfasst, wird noch
  ein gesonderter Kurztitel gebraucht: \texttt{\textbackslash title[<Kurztitel mit weniger oder gleich 65 Zeichen>]\{<Langtitel mit mehr als 65 Zeichen>\}}}

\todo{Der Umfang muss genau 10 Seiten sein (inklusive Literatur und Lebenslauf).}

\begin{abstract}
  \todo{
    Der Abstract gibt einen kurzer Überblick über die Kurzfassung mit
    ca. 10 Zeilen.} \blindtext
\end{abstract}

\section{Einleitung}

\blindtext

\section{Demonstrationen}
\label{sec:demos}
Das Symbol für Potenzmengen ($\powerset$) wird korrekt angezeigt.
Es ist kein Weierstraß-p ($\wp$) mehr.

Spitze Klammen können direkt eingegeben werden: <test />

Hier eine kleine Demonstration von \href{https://www.ctan.org/pkg/microtype}{microtype}:
\blindtext

\subsection{Literaturverzeichnis}

\todo{Das Literaturverzeichnis muss eine Referenz auf die publizierte
Dissertation \mbox{\cite{MeineDiss}} enthalten.}

Der letzte Abschnitt zeigt ein beispielhaftes Literaturverzeichnis für Bücher mit einem Autor \cite{Ez10} und zwei AutorInnen \cite{AB00}, einem Beitrag in Proceedings mit drei AutorInnen \cite{ABC01}, einem Beitrag in einem LNI Band mit mehr als drei AutorInnen \cite{Az09}, zwei Bücher mit den jeweils selben vier AutorInnen im selben Erscheinungsjahr \cite{Wa14} und \cite{Wa14b}, ein Journal \cite{Gl06}, eine Website \cite{GI19} bzw.\ anderweitige Literatur ohne konkrete AutorInnenschaft \cite{XX14}.
Es wird biblatex verwendet, da es UTF8 sauber unterstützt und \href{https://github.com/gi-ev/LNI/issues/5}{im Gegensatz zu lni.bst} keine Fehler beim bibtexen auftreten.

Referenzen sollten nicht direkt als Subjekt eingebunden werden, sondern immer nur durch Authorenanganben:
Beispiel: \Citet{AB00} geben ein Beispiel, aber auch \citet{Az09}.
Hinweis: Großes C bei \texttt{Citet}, wenn es am Satzanfang steht. Dies ist analog zu \texttt{Cref}.

Formatierung und Abkürzungen werden für die Referenzen \texttt{book}, \texttt{inbook}, \texttt{proceedings}, \texttt{inproceedings}, \texttt{article}, \texttt{online} und \texttt{misc} automatisch vorgenommen.
Mögliche Felder für Referenzen können der Beispieldatei \texttt{gidiss-kurzfassung.bib} entnommen werden.
Andere Referenzen sowie Felder müssen allenfalls nachträglich angepasst werden.

\subsection{Abbildungen}
\Cref{fig:demo} zeigt eine Abbildung.

\begin{figure}
  \centering
  \includegraphics[width=.55\textwidth]{example-image}
  \caption{Demographik}
  \label{fig:demo}
\end{figure}

\subsection{Tabellen}
\Cref{tab:demo} zeigt eine Tabelle.

\begin{table}
\centering
\begin{tabular}{lll}
\toprule
Überschriftsebenen & Beispiel & Schriftgröße und -art \\
\midrule
Titel (linksbündig) & Der Titel \ldots & 14 pt, Fett\\
Überschrift 1 & 1 Einleitung & 12 pt, Fett\\
Überschrift 2 & 2.1 Titel & 10 pt, Fett\\
\bottomrule
\end{tabular}
\caption{Die Überschriftsarten}
\label{tab:demo}
\end{table}

\subsection{Programmtext}
Die LNI-Formatvorlage verlangt die Einrückung von Listings vom linken Rand.
In der \texttt{lni}-Dokumentenklasse ist dies für die \texttt{verbatim}-Umgebung realisiert.

\begin{verbatim}
public class Hello {
    public static void main (String[] args) {
        System.out.println("Hello World!");
    }
}
\end{verbatim}

Alternativ kann auch die \texttt{lstlisting}-Umgebung verwendet werden.

\Cref{L1} zeigt uns ein Beispiel, das mit Hilfe der \texttt{lstlisting}-Umgebung realisiert ist.

\begin{lstlisting}[caption={Beschreibung}, label=L1, language=Java]
public class Hello {
    public static void main (String[] args) {
        System.out.println("Hello World!");
    }
}
\end{lstlisting}

\subsection{Formeln und Gleichungen}

Die korrekte Einrückung und Nummerierung für Formeln ist bei den Umgebungen
\texttt{equation} und \texttt{align} gewährleistet.

\begin{equation}
  1=4-3
\end{equation}
und
\begin{align}
  2&=7-5\\
  3&=2-1
\end{align}

\section{Schlussbemerkungen}

\blindtext

%% \bibliography{gidiss-kurzfassung} ist hier nicht erlaubt: biblatex erwartet dies bei der Präambel
%% Starten Sie "biber paper", um eine Bibliographie zu erzeugen.
\printbibliography

\todo{Den Schluss des Beitrags (hinter dem Literaturverzeichnis)
bildet ein kurzer (ca. 1/3 Seite) Lebenslauf des Autors. Üblicherweise
mit einem Portraitbild (4:5), auf der linken Seite, vom Text umflossen.}
\begin{wrapfigure}{l}{.25\textwidth}
\includegraphics[angle=90,origin=c,width=.25\textwidth]{example-image-empty}
\end{wrapfigure}
\noindent\textbf{Vorname Nachname} wurde am 1. Januar 1970
geboren. \blindtext

\end{document}
